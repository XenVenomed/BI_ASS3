\documentclass[sigconf]{acmart}

\AtBeginDocument{ \providecommand\BibTeX{ Bib\TeX } }
\setcopyright{acmlicensed}
\copyrightyear{2025}
\acmYear{2025}
\acmDOI{XXXXXXX.XXXXXXX}

\acmConference[BI 2025]{Business Intelligence Final Report}{-}{-}

\begin{document}

\title{BI2025 Final Report - Group 74}
%% ---Authors: Dynamically added ---
\author{Avelardo Ramirez} \authornote{Student A, Matr.Nr.: 12435655} \affiliation{\institution{TU Wien} \country{Austria}}\author{Agon Sylejmani} \authornote{Student B, Matr.Nr.: 01556207} \affiliation{\institution{TU Wien} \country{Austria}}

\begin{abstract}
  this report documents the complete crisp-dm cycle for group 74 analyzing obesity levels.
  it covers business and data understanding, preparation, modeling using random forest,
  extensive evaluation including bias analysis, and deployment recommendations.
\end{abstract}

\ccsdesc[500]{computing methodologies~machine learning}
\keywords{crisp-dm, provenance, random forest, obesity classification, bias analysis}

\maketitle

%% --- 1. BUSINESS UNDERSTANDING ---
\section{Business Understanding}
\subsection{Data Source and Scenario} **Data Source:**
The dataset contains 2,111 records from individuals in Mexico, Peru, and Colombia,
collected to estimate obesity levels based on eating habits and physical condition.
The data includes 17 attributes covering demographics (age, gender, height, weight),
eating habits (high-calorie food consumption, vegetable consumption, number of meals,
water intake, alcohol consumption), and physical activity patterns (exercise frequency,
technology usage time, transportation mode).

**Business Scenario:**
A public health agency in Latin America aims to combat the rising obesity epidemic
by implementing targeted intervention programs. The agency needs an automated system
to classify individuals into obesity risk categories based on their lifestyle and
physical characteristics. This classification will enable:
1. Early identification of at-risk populations
2. Personalized health recommendations
3. Resource allocation for intervention programs
4. Monitoring of public health trends over time

The system will be deployed as a web-based screening tool accessible to healthcare
providers and wellness centers across Mexico, Peru, and Colombia.
\subsection{Business Objectives} The primary business objectives are:

1. **Reduce Obesity Prevalence:** Support public health initiatives aimed at
   reducing obesity rates.

2. **Enable Targeted Interventions:** Provide healthcare professionals with an
   accurate classification tool that identifies specific obesity risk categories,
   allowing for customized intervention strategies for each risk group.

3. **Improve Resource Allocation:** Help health agencies allocate resources
   efficiently by identifying geographic regions and demographic groups with
   highest obesity risk.

4. **Support Preventive Care:** Enable early detection of obesity risk before
   severe health complications develop.

5. **Provide Data-Driven Insights:** Generate actionable insights about the
   relationship between lifestyle factors and obesity levels to inform public health policy.
\subsection{Business Success Criteria} The success of this business initiative will be measured by:

1. **Adoption Rate:** Achieve 70\% adoption rate among targeted healthcare
   facilities within the first year of deployment.

2. **Intervention Effectiveness:** Demonstrate that individuals identified as
   high-risk who receive targeted interventions show measurable improvement
   (BMI reduction of at least 2 points).

3. **Cost-Effectiveness:** Reduce overall healthcare costs related to obesity
   complications by 15\% over 3 years through early intervention.

4. **User Satisfaction:** Achieve at least 80\% satisfaction rating from
   healthcare providers using the tool, measured through user surveys.

5. **Coverage:** Successfully screen at least 50,000 individuals within the
   first year across the three target countries.

6. **Actionability:** Ensure that 90\% of high-risk classifications result in
   documented intervention actions by healthcare providers.
\subsection{Data Mining Goals} The specific data mining goals are:

1. **Multi-class Classification:** Build a robust classifier that accurately
   predicts obesity levels across all 7 categories:
   - Insufficient Weight
   - Normal Weight
   - Overweight Level I
   - Overweight Level II
   - Obesity Type I
   - Obesity Type II
   - Obesity Type III

2. **Feature Importance Analysis:** Identify which eating habits and physical
   activity factors are most predictive of obesity levels to guide intervention
   design.

3. **Balanced Performance:** Achieve strong performance across all obesity
   categories, not just the majority classes, ensuring reliable predictions
   for minority obesity types.

4. **Generalization:** Develop a model that generalizes well across different
   demographic groups (age ranges, genders) and geographic regions.

5. **Interpretability:** Create a model whose predictions can be explained to
   healthcare providers and patients, supporting trust and actionable insights.

%% --- 2. DATA UNDERSTANDING ---
\section{Data Understanding}
\subsection{Dataset Overview} Dataset containing obesity levels based on eating habits and physical condition from individuals in Mexico, Peru, and Colombia. Contains 2,111 instances with 17 attributes including demographic, lifestyle, and physical measurements.
\subsection{Attribute Analysis}
\begin{table*}[t]
  \caption{dataset features}
  \small
  \begin{tabular}{p{0.18\linewidth}p{0.12\linewidth}p{0.62\linewidth}}
    \toprule \textbf{feature name} & \textbf{data type} & \textbf{description} \\ \midrule
    Age & integer> & Age of the individual in years \\
    CAEC & string> & Consumption of food between meals (no/Sometimes/Frequently/Always) \\
    CALC & string> & Frequency of alcohol consumption (no/Sometimes/Frequently/Always) \\
    CH2O & double> & Daily water consumption in liters \\
    FAF & double> & Physical activity frequency per week (0-3 scale, where 0=no activity, 3=4+ days/week) \\
    FAVC & string> & Frequent consumption of high caloric food (yes/no) \\
    FCVC & double> & Frequency of vegetable consumption (1-3 scale, where 1=never, 2=sometimes, 3=always) \\
    Gender & string> & Gender of the individual (Female/Male) \\
    Height & double> & Height of the individual in meters \\
    MTRANS & string> & Mode of transportation usually used (Automobile/Motorbike/Bike/Public\_Transportation/Walking) \\
    NCP & double> & Number of main meals consumed per day (typically 1-4) \\
    NObeyesdad & string> & Obesity level classification: Insufficient\_Weight, Normal\_Weight, Overweight\_Level\_I, Overweight\_Level\_II, Obesity\_Type\_I, Obesity\_Type\_II, Obesity\_Type\_III \\
    SCC & string> & Monitors calorie consumption (yes/no) \\
    SMOKE & string> & Whether the individual smokes (yes/no) \\
    TUE & double> & Time using technology devices (computer, smartphone, TV, etc.) in hours per day \\
    Weight & double> & Weight of the individual in kilograms \\
    family\_history\_with\_overweight & string> & Whether the individual has family members with overweight (yes/no) \\
    \bottomrule
  \end{tabular}
\end{table*}
\subsection{Statistical Properties} Task 2b: Statistical Properties and Correlations
Key Findings:
- Imbalance in class distribution: Obesity\_Type\_I (25.2\%), Normal\_Weight (21.5\%),
  Overweight\_Level\_II (13.6\%), Overweight\_Level\_I (13.5\%), Obesity\_Type\_II (13.5\%),
  Obesity\_Type\_III (11.3\%), Insufficient\_Weight (1.4\%)
- No strong correlations (|r| > 0.5) found between numeric features
- Moderate correlations observed:
  * Height-Weight (r=0.463): expected physiological relationship
  * Height-FAF (r=0.295): taller individuals slightly more active
  * Height-NCP (r=0.244): taller individuals eat more meals
- Skewness analysis:
  * Age: 1.529 (right-skewed) - dataset contains more younger individuals
  * NCP: -1.107 (left-skewed) - most people eat 3-4 main meals
  * TUE: 0.619 (right-skewed) - most have low tech use, some high users
  * Other features approximately symmetric
- Descriptive statistics show reasonable ranges for all numeric features
\subsection{Data Quality} Task 2c: Data Quality Assessment
Key Findings:
1. Missing Values: None detected
2. Duplicate Rows: 24 duplicates found
3. Outliers (IQR method):
   - Age: 168 outliers (7.96\%) - elderly individuals above 35 years
   - NCP: 579 outliers (27.43\%) - individuals eating <2.15 or >3.51 meals/day
   - Height: 1 outlier (0.05\%) - likely data entry error or very tall individual
   - Weight: 1 outlier (0.05\%) - likely very heavy individual or error
   - Other features: no outliers detected
4. Plausibility:
   - Age: [14-61] years ✓ realistic range
   - Height: [1.45-1.98]m ✓ realistic range
   - Weight: [39-173]kg ✓ realistic range
   - All values fall within biologically plausible ranges
5. Categorical Consistency: All categorical variables have expected, consistent values
Data Quality Summary: High quality dataset with minimal issues.
\subsection{visual exploration}

\begin{figure}[h]
    \centering
    \includegraphics[width=0.8\linewidth]{}
    \caption{visual analysis of obesity factors.}
    \label{fig:viz_2d}
\end{figure}
\subsection{Ethical Sensitivity} Task 2e: Ethical Sensitivity Assessment
Potentially Sensitive Attributes Identified:
1. Gender (Female/Male)
   - Protected characteristic under most anti-discrimination laws
   - Risk: Model could learn gender stereotypes about eating habits or body composition
   - Mitigation: Ensure equal performance across genders, test for disparate impact

2. Age (14-61 years range)
   - Protected in employment and some health contexts
   - Risk: Age-based discrimination in health interventions
   - Younger individuals (14-17) are minors,this requires special consideration

3. Family History with Overweight (yes/no)
   - Potentially sensitive genetic/family information
   - Risk: Could be used to discriminate based on genetic predisposition
   - Not typically protected but ethically sensitive

Underrepresented Groups:
1. Insufficient Weight class: Only 1.4\% of dataset
   - Risk: Model may perform poorly on this minority class
   - Action: Consider oversampling or stratified evaluation

2. Extreme age groups (14-17, 55+): Underrepresented
   - Risk: Model may not generalize well to these age ranges
   - Action: Ensure test set includes these groups for validation

Class Imbalance Analysis:
- Obesity Type I: 25.2\% (largest class)
- Insufficient Weight: 1.4\% (smallest class)
- 18x difference between largest and smallest class
- Recommendation: Use macro-averaged metrics and per-class evaluation

%% --- 3. DATA PREPARATION ---
\section{Data Preparation}
\subsection{Applied Actions} **3a. Applied Pre-processing Actions:**
1. **Cleaning:** Deduplicated dataset (24 duplicates removed).
2. **Feature Engineering:** Calculated 'BMI' and created 'Age\_Group' bins.
3. **Encoding:** Applied LabelEncoding (Target), OrdinalEncoding (CAEC/CALC), and OneHotEncoding (MTRANS).
4. **Scaling:** Standardized all continuous features (Mean=0, Std=1) to ensure equal model weighting.
\subsection{Rejected Steps} 
\subsection{Derived Attributes} Analysis of Derived Attributes (Feature Engineering):

We analyzed which new features could help the model learn better patterns:

1.  BMI (Body Mass Index) - [APPLIED]:
    * We calculated this using Weight / (Height\textasciicircum{}2).
    * Why: Even though the model has Height and Weight, providing the explicit BMI ratio helps decision trees make cleaner splits. It is historically the strongest predictor for obesity.

2.  Age Grouping (Binning) - [APPLIED]:
    * We converted the continuous 'Age' into categories (Youth, Adult, Senior).
    * Why: Lifestyle habits change with life stages. A 20-year-old and a 50-year-old might have the same weight but very different health risks. This helps the model find those non-linear patterns.

3.  "Sedentary Ratio" (TUE / FAF) - [REJECTED]:
    * Idea: Create a ratio of "Time on Technology" divided by "Physical Activity" to quantify a sedentary lifestyle.
    * Problem: Many participants have FAF = 0 (no activity). This causes division-by-zero errors. Also, combining them might hide the specific impact of just "sitting too much" vs "not moving enough". We kept them separate.

4.  Healthy Diet Score - [REJECTED]:
    * Idea: Summing up vegetable intake and water, subtracting junk food.
    * Reason: Information Loss. A person who eats lots of veggies AND lots of junk food is different from someone who eats neither. The model needs to see the individual habits to classify correctly.

%% --- 4. MODELING ---
\section{Modeling}
\subsection{Algorithm Selection}
for the modeling phase, we selected the random forest classifier.
our dataset has 7 different obesity levels as target classes, and
about 77 percent of the data is synthetic generated by smote.
random forest is an ensemble method that is very robust against
overfitting on these synthetic patterns. it also handles the
outliers we found in the age and ncp attributes much better
than linear models. another advantage is that it provides
feature importance, which is great for our public health scenario
to see which habits have the most impact on obesity.
\subsection{Hyperparameter Identification and Tuning}
we identified several hyper-parameters for the random forest classifier,
including n\_estimators, max\_depth, and min\_samples\_split.
for our experiments, we select 'max\_depth' as the primary parameter for tuning.
this choice is justified because it directly controls the complexity
of the individual trees. since the dataset contains 77 percent synthetic
records generated by smote, there is a high risk of the model learning
noise or artificial patterns. tuning max\_depth allows us to find the
optimal balance between bias and variance and ensures better
generalization for real-world obesity screening.
\textit{Note: tuning results visualized in the attached plots.}
\subsection{Data Splitting Strategy}
we implemented a stratified 60/20/20 split to handle the 7 obesity classes.
stratification ensures that the distribution of obesity levels remains
consistent across train, validation, and test sets.
we used a fixed random seed (42) to ensure reproducibility as
required by the assignment.
\subsection{Final Model Retraining}
retrained the final random forest model on the complete
training and validation data using max\_depth=15.

%% --- 5. EVALUATION ---
\section{Evaluation}
\subsection{Final Test Performance}
the final random forest model achieved a test accuracy of 0.9856.
this meets our data mining success criteria.
we compared it against kaggle benchmarks (95-99\%) and a random baseline.
\textbf{Resulting Test Accuracy:} 0.9856459330143541
\subsection{Bias and Fairness Analysis}
accuracy female: 0.9798, accuracy male: 0.9909

%% --- 6. DEPLOYMENT ---
\section{Deployment}
\subsection{Recommendations}
our final random forest model achieved a test accuracy of over 90 percent,
which successfully fulfills the business success criteria defined during
the first phase of crisp-dm. since the performance is consistently high
across all seven obesity levels, the tool provides a solid foundation for
public health agencies to identify at-risk populations early. we recommend
a hybrid deployment strategy where the model acts as a preliminary
screening tool for clinics in mexico, peru, and colombia. however,
automated results must always be verified by healthcare professionals to
avoid misdiagnosis, especially for minority classes or unusual lifestyle patterns.
\subsection{Ethical Risks}
the most significant ethical concern is the use of 77 percent synthetic
data generated via smote. while this helps with class balance, it might
introduce artificial patterns that do not perfectly reflect the biological
diversity of real patients. furthermore, the model is geographically
limited to three latin american countries, which could lead to bias if
applied to other regions with different dietary cultures. we must also
ensure that the classification does not lead to patient stigmatization
or discrimination in insurance contexts, requiring strict data
privacy protocols and human oversight.
\subsection{Monitoring and Maintenance}
to maintain model reliability, we propose a two-tier monitoring plan.
first, we must monitor for data drift, specifically tracking if the
distribution of eating habits or transportation modes in new patients
shifts significantly from our training set. second, we define a
performance trigger: we will perform regular audits against manual medical
labels. if the classification accuracy for any specific subgroup or
the overall population drops below 85 percent, it will trigger an
immediate review and potential retraining of the model.
\subsection{Reproducibility Reflection}
reproducibility of our experiment is high because we used fixed random
seeds (42) for all data splits and training runs. every processing step,
from initial data cleaning and bmi calculation to the final hyperparameter
tuning of the random forest, is documented within this provenance
knowledge graph. a remaining risk for reproducibility lies in the
dependency on specific library versions for scaling and preprocessing,
which must be documented clearly in the final report to ensure
consistent results across different environments.

\section{Conclusion}
the project successfully demonstrated the application of the crisp-dm process
to classify obesity levels with high accuracy. the provenance logging
ensures full transparency of all modeling and evaluation decisions.

\end{document}
